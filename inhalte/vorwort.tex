%%%%%%%%%%%%%%%%%%%%%%%%%%%%%%%%%%%%%%%%%%%%%%%%%%%%%%%%%%%%%%%%
%% Vorlage fuer akademische Arbeiten, Gestaltungsrichtlinien der
%% Medienwissenschaft an der Universitaet Regensburg
%% 
%% Gutes Gelingen  --  Juli 2008
%% Christoph Mandl und Christoph Pfeiffer
%% http://www-mw.uni-r.de/studium/materialien 
%%
%% CC/BY-SA/3.0 - http://creativecommons.org/licenses/by-sa/3.0/
%%
%%%%%%%%%%%%%%%%%%%%%%%%%%%%%%%%%%%%%%%%%%%%%%%%%%%%%%%%%%%%%%%%

\cleardoublepage
\phantomsection

%\thispagestyle{section}
\abstand
\addcontentsline{toc}{section}{Vorwort}

\noindent \textbf{\textsf{\large }} \newline

\begin{spacing}{1.4}
\begin{figure}
\centering
\begin{minipage}{56mm}
\vspace{2cm}
\lettrine[lines=2, lhang=0.25, loversize=0.25, findent=3pt, nindent=0pt]{F}{�r} die Erstellung schriftlicher Arbeiten in der Regensburger Medienwissenschaft soll dieses kurze Dokument als Leitfaden dienen. Es behandelt sowohl allgemeine typografische Gesichtspunkte wie Lesbarkeit und �sthetik als auch formale Kriterien wie Aufbau der Arbeit und empfohlene Zitierweise. Die Richtlinien sind f�r Arbeiten in der Medienwissenschaft verbindlich und k�nnen sich im Einzelnen von Vorgaben in anderen F�chern unterscheiden.
\end{minipage}
\end{figure}
\end{spacing}